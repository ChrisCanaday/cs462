\documentclass{article}
\usepackage{amsmath}
\usepackage{enumerate}
\usepackage{textcomp}

\title{Homework 1}
\author{Christopher Canaday}
\date{\today}

\begin{document}
    \maketitle
    \textbf{Problem 1:} Let the strong speedup of a problem of size n that is computationally solved using \textit{P} processors
by a parallel algorithm in time $T(n, P)$ be defined by:
    \begin{equation}
        S(p) = \frac{T(n,1)}{T(n,P)}
    \end{equation}
    and the efficiency of the same parallel algorithm be defined by
    \begin{equation}
        E(p) = \frac{S(p)}{P}
    \end{equation}
    Suppose a parallel algorithm has a parallel runtime complexity of:
    \begin{equation}
        T(n,p) = \Theta(\frac{n^2}{P} + \sqrt{n}) \phantom{text}\text{for } P \leq n^2
    \end{equation}

    \begin{enumerate}[i.]
        \item What is the execution time of this parallel algorithm on a single processor (we assume a single
        process is mapped to a single processor)?
            \begin{enumerate}
                \item The execution time of the parallel algorithm with a single processor would be:
                \begin{equation}
                    T(n,1) = \Theta(n^2 + \sqrt{n})
                \end{equation}
            \end{enumerate}
        \item In terms of \textit{n}, at what processor count, $P_{\max}$, will the algorithm achieve maximum parallel
        speedup?
            \begin{enumerate}
                \item The algorithm will achieve maximum parallel speedup when $P_{\max} = n^2$. The runtime complexity would be:
                    \begin{align*}
                        T(n, P_{\max}) & = \Theta(\frac{n^2}{P_{\max}} + \sqrt{n}) \phantom{text}\text{for } P_{\max} = n^2 \\
                        & = \Theta(\frac{n^2}{n^2} + \sqrt{n}) \\
                        & = \Theta(1 + \sqrt{n}) \\
                        & = \Theta(\sqrt{n})
                    \end{align*}
                    The speedup $S(P_{\max})$ would be:
                    \begin{align*}
                        S(P_{\max}) & = \frac{T(n,1)}{T(n,P_{\max})} \\
                        & = \frac{\Theta(n^2 + \sqrt{n})}{\Theta(\sqrt{n})} \\
                        & = \Theta(\frac{n^2 + \sqrt{n}}{\sqrt{n}}) \\
                        & = \Theta(\frac{n^2 + \sqrt{n}}{\sqrt{n}} * \frac{\sqrt{n}}{\sqrt{n}} ) \\
                        & = \Theta(\frac{n^2\sqrt{n}  + n}{n}) \\
                        & = \Theta(n\sqrt{n} + 1) \\
                        & = \Theta(n\sqrt{n})
                    \end{align*}
            \end{enumerate}
        \item What is the parallel efficiency achieved when $P_{\max}$ processors are used?
            \begin{enumerate}
                \item The parallel efficiency is:
                \begin{align*}
                    E({P_{\max}}) & = \frac{S(P_{\max})}{P_{\max}} \\
                    & = \frac{n\sqrt{n}}{n^2} \\
                    & = \frac{\sqrt{n}}{n}
                \end{align*}
            \end{enumerate}
        \item Maximum parallel efficiency is achieved when $E(P)\approx1$. At what processor count will this
        algorithm achieve its maximum parallel efficiency?
            \begin{enumerate}
                \item The maximum efficiency would be with 1 processor.
                    \begin{align*}
                        E(P) & = \frac{S(P)}{P} \phantom{text}\text{for } S(P)=\frac{T(n,1)}{T(n,P)} \text{ and } E(P) \approx 1 \\
                        1 & \approx \frac{\frac{T(n,1)}{T(n,P)}}{P} \\
                        1 & \approx \frac{\frac{n^2+\sqrt{n}}{\frac{n^2}{p}+\sqrt{n}}}{P} \\
                        1 & \approx \frac{n^2+\sqrt{n}}{\frac{n^2}{P}+\sqrt{n}} * \frac{1}{P} \\
                        1 & \approx \frac{n^2+\sqrt{n}}{n^2+P\sqrt{n}} \\
                        n^2+P\sqrt{n} & \approx n^2+\sqrt{n} \\
                        P\sqrt{n} & \approx \sqrt{n} \\
                        P & \approx 1
                    \end{align*}
            \end{enumerate}
    \end{enumerate}

    \textbf{Problem 2:} The following is a polynomial of a single unknown variable x and degree n\textminus1 where $a_{0}$, $a_{1}$, · · · , $a_{n-1}$
    are known constants.

    \begin{equation}
        P(x) = a_{0} + a_{1}x + a_{2}x^2 + a_{3}x^3 + \cdots + a_{n-1}x_{n-1}
    \end{equation}

    The goal here is to compute the value of the polynomial at a given value of x, say, x\textsubscript{0}, that is, evaluate $P(x_{0})$.
    Describe step-by-step a parallel algorithm to compute $P(x_{0})$. Assume that n = k\textsubscript{1}p and p = 2\textsuperscript{k\textsubscript{{2}}} where k\textsubscript{1}
    and k\textsubscript{2} are both positive integers greater than 1. What are the parallel computation and communication
    costs of your algorithm.

    \begin{enumerate}[i.]
        \item The problem is perfect for Horner's Method. The algorithm steps are:
            \begin{enumerate}
                \item Partition a\textsubscript{i} in parts and assign them to p processes.
                \item Broadcast the value of x\textsubscript{0}.
                \item Do parallel prefix with the multiplication operation.
                \item Locally multiply the partial products with the right coefficient.
                \item Locally add all the products.
                \item Do a reduce to add all the local products together.
            \end{enumerate}
        \item The computation cost is:
            \begin{equation}
                {T(n,p)}_{comp} = \Theta(C_{op} * (\frac{n}{p} + \log_{2} p))
            \end{equation}
            Where C\textsubscript{op} is the cost of performing the multiplication operation. 
            There are $\frac{n}{p}$ computations in each process and during the parallel prefix you will additionally multiply
            $\log_{2}p$ times. There are also $\frac{n}{p}$ additions in each process, but this gets consumed by the $\Theta$.
        \item The communication cost is:
            \begin{equation}
                {T(n,p)}_{comm} = C_{reduce} + C_{parallelPrefix} + C_{broadcast} + C_{personalizedSend}
            \end{equation}

            Considering that p is guaranteed to be a power of two I am going to assume that the network is in a HyperCube configuration.
            With a HyperCube Configuration the cost of a reduce would be:
            \begin{equation}
                C_{reduce}^{HyperCube} = \Theta((\tau + \mu m_{0})*\log_{2}p)
            \end{equation}
            Where m\textsubscript{0} is the message size. In this case the message size would be one number from each process. The cost for the parrallel prefix would be:
            \begin{equation}
                C_{parallelPrefix}^{HyperCube} = \Theta((\tau + \mu m_{0})*\log_{2}p)
            \end{equation}
            Where m\textsubscript{0} is a number being sent to the process to the right. The $\log_{2}p$ comes from the number of times this occurs during the parallel prefix.
            The cost for the broadcast of x is:
            \begin{equation}
                C_{broadcast}^{HyperCube} = \Theta((\tau + \mu m_{0})*\log_{2}p)
            \end{equation}
            This operation is the same as reduce, but in reverse so it has the same cost. To send the a's we have to do a personalized send to give processes the correct value. The cost for a personalized send is:
            \begin{equation}
                C_{personalizedSend}^{HyperCube} = \Theta(\tau \log_{2}{p} + \mu m_{1}{p})
            \end{equation}
            Here m\textsubscript{1} is the size of $\frac{n}{p}$ numbers. In other words.
            \begin{equation}
                m_{1} = \frac{n}{p}*m_{0}
            \end{equation}
            Thus,
            \begin{equation}
                C_{personalizedSend}^{HyperCube} = \Theta(\tau \log_{2}{p} + \mu m_{0}{n})
            \end{equation}
            So the communication cost assuming a HyperCube network configuration is:
            \begin{multline*}
                {T(n,p)}_{comm}^{HyperCube} = \Theta((\tau + \mu m_{0})*\log_{2}p) \\
                + \Theta((\tau + \mu m_{0})*\log_{2}p) \\
                + \Theta((\tau + \mu m_{0})*\log_{2}p) \\
                + \Theta(\tau \log_{2}{p} + \mu m_{0}{n}) \\
            \end{multline*}
            Which simplified is:
            \begin{align*}
                {T(n,p)}_{comm}^{HyperCube} &= \Theta((\tau + \mu m_{0})*\log_{2}p + \tau \log_{2}{p} + \mu m_{0}{n}) \\
                &= \Theta(\tau \log_{2}p + \mu m_{0}\log_{2}p + \mu m_{0}{n}) \\
                &= \Theta(\tau \log_{2}p + \mu m_{0}(\log_{2}p + {n})) \\
            \end{align*}
            So, the total cost is:
            \begin{equation}
                {T(n,p)}_{total}^{HyperCube} = \Theta(C_{op} * (\frac{n}{p} + \log_{2} p) + \tau \log_{2}p + \mu m_{0}(\log_{2}p + {n}))
            \end{equation}

            Where C\textsubscript{op} is the cost of the multiplication operation, m\textsubscript{0} is the size of a number, and $\mu$ is $\frac{1}{transferRate}$.
    \end{enumerate}

\end{document}