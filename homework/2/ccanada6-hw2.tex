\documentclass{article}
\usepackage{amsmath}
\usepackage{enumerate}
\usepackage{textcomp}
\usepackage{tikz}
\usetikzlibrary{shapes.multipart}
\usetikzlibrary{calc}
\newcounter{nodeidx}
\setcounter{nodeidx}{1}

\tikzset{block/.style={
        font=\sffamily,
        draw=black,
        thin,
        fill=pink!50,
        rectangle split,
        rectangle split horizontal,
        rectangle split parts=#1,
        outer sep=0pt},
        %
        gblock/.style={
            block,
            rectangle split parts=#1,
            fill=green!30}
        }

\title{Homework 2}
\author{Christopher Canaday}
\date{\today}

\begin{document}
    \maketitle
    \textbf{Problem 1:} (5 points) Describe the communication cost model used in class for complexity analyses of all
    parallel algorithms. Be sure to describe all the assumptions underlying the cost model.
    \begin{enumerate}
        \item A pair of con nected processors can send message of size m to each other simultaneously in time.
        \item A processor can send a message on only 1 of its links at a time.
        \item A processor can receive a message on only 1 of its links at a time.
        \item A processor can receive a message while sending another message at the same time on the same or a different link.
        \item Processors use store-forward routing for communication.
    \end{enumerate}

    \textbf{Problem 2:} Assume that a pool of 4 processors {$p_{0}$, $p_{1}$, $p_{2}$, $p_{3}$} is available.
    \begin{enumerate}[i.]
    \item (5 points) If $p_{0}$ wants the value of a local variable a to be known to all the 4 processors, which collective
    communication call needs to be executed?
    \begin{enumerate}
        \item You would use the broadcast communication call. MPI\_Bcast.
    \end{enumerate}

    \item (5 points) If $p_{0}$ wants to distribute a locally defined array [$d_{0}$, $d_{1}$, $d_{2}$, $d_{3}$] amongst the 4 processors such
that $d_{0}$ is sent to $p_{0}$, $d_{1}$ is sent to $p_{1}$, $d_{2}$ is sent to $p_{2}$ and $d_{3}$ is sent to $p_{3}$, which collective call needs to be
executed.
    \begin{enumerate}
        \item You would use the scatter communication call. MPI\_Scatter.
    \end{enumerate}
    \end{enumerate}
    
    \textbf{Problem 3:}
    \begin{enumerate}[i.]
    \item (5 points) Give an example of a bitonic sequence of 8 integers.
    \begin{enumerate}
        \item 2, 6, 7, 9, 11, 10, 5, 3
    \end{enumerate}

    \item (5 points) Sort this bitonic sequence of 8 integers using bitonic splits. Show each bitonic splitting step
    during the sorting algorithm clearly.

    \begin{center}

        \def\lvld{1}                  % Choose level distance
        \pgfmathsetmacro\shft{-4*\lvld} % Calculate the yshift for the green tree

        \begin{tikzpicture}[level distance=\lvld cm,
            level 1/.style={sibling distance=4cm},
            level 2/.style={sibling distance=2cm},
            level 3/.style={sibling distance=1cm},
            edgedown/.style={edge from parent/.style={draw=red,thick,-latex}},
            edgeup/.style={edge from parent/.style={draw=green!50!black,thick,latex-}}
            ]

        % PINK TREE

        \node[block=8] (A) {2 \nodepart{two} 6 \nodepart{three} 7 \nodepart{four} 9 \nodepart{five} 11 \nodepart{six}10 \nodepart{seven}5 \nodepart{eight}3}
        [grow=down,edgedown]
        child {node[block=4] (B1) {2 \nodepart{two} 6 \nodepart{three} 5 \nodepart{four} 3}
            child {node[block=2] (C1) {2 \nodepart{two} 3}
                child {node[block=1] (D1) {2}}
                child {node[block=1] (D2) {3}}
                }
            child {node[block=2] (C2) {5 \nodepart{two} 6}
                child {node[block=1] (D3) {5}}
                child {node[block=1] (D4) {6}}
                }
            }
        child {node[block=4] (B2) {11 \nodepart{two} 10 \nodepart{three} 7 \nodepart{four} 9}
            child {node[block=2] (C3) {7 \nodepart{two} 9}
                child {node[block=1] (D5) {7}}
                child {node[block=1] (D6) {9}}
            }
            child {node[block=2] (C4) {11 \nodepart{two} 10}
                child {node[block=1] (D7) {10}}
                child {node[block=1] (D8) {11}}
            }
        };
        \end{tikzpicture}

    \textit{Figured out how to make this graph here: https://tex.stackexchange.com/questions/592155/how-to-draw-a-merge-sort-algorithm-figure}
        \end{center}

    \end{enumerate}


    \textbf{Problem 4:} Matrix-Vector Multiplication
    \begin{enumerate}[i.]
    \item (5 points) The parallel complexity of a matrix-vector algorithm can be shown to be:
    \begin{equation}
        T(n,p) = O(\frac{n^2}{p} + \tau\log{p}+\mu{n})
    \end{equation}
    where the symbols have their usual meanings. What is the cost of this parallel algorithm where cost is
defined as above.

    \item (5 points) (10 points) \textit{Cost-optimality} of this algorithm will be achieved when (a) $p = O(\sqrt{n})$, (b) $p = O(n)$ (c),
    $p = O(n^2)$, or (d) $p = O(n^3)$. Choose the correct answer from the four given options and analytically
    justify your choice.
    \end{enumerate}

    \textbf{Cost Optimality} \\
    The cost of solving a problem on a parallel system is defined as the product of the parallel runtime and
    the number of processing elements used. In other words, it is the sum of the time that each processing
    element spends solving the problem. The cost of solving a problem on a single processing element is the
    execution time of the fastest known sequential algorithm. A parallel algorithm is said to be cost-optimal
    if the cost of solving the parallel algorithm has the same asymptotic growth (in $\Theta$ terms) as a function of
    the input size as the fastest known sequential algorithm on a single processor. Since efficiency is the ratio
    of sequential cost to parallel cost, a cost-optimal parallel system has an efficiency of $\Theta$\(1\) (that is, a constant). \\


    \textbf{Problem 5:} Matrix-Matrix Multiplication
    \begin{enumerate}[i.]
    \item (5 points) The parallel complexity of a simple matrix-matrix algorithm can be shown to be:
    \begin{equation}
        T(n,p) = O(\frac{n^3}{p} + \tau\log{p}+2\mu\frac{n^2}{\sqrt{p}})
    \end{equation}
    where the symbols once againhave their usual meanings. What is the cost of this parallel algorithm
where cost is again defined as above.

    \item (5 points) (10 points) \textit{Cost-optimality} of this algorithm will be achieved when (a) $p = O(\sqrt{n})$, (b) $p = O(n)$ (c),
    $p = O(n^2)$, or (d) $p = O(n^3)$. Choose the correct answer from the four given options and analytically
    justify your choice.
    \end{enumerate}

Note that you will need to know the complexities of the fastest sequential algorithms for matrix-vector and
matrix-matrix multiplications, both of which were covered in class

    
\end{document}